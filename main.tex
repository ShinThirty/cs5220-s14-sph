\documentclass[10pt, leqno]{article} %% use to set typesize 
\usepackage{amsfonts, amsmath}
\usepackage{fancyhdr}
\usepackage[
  letterpaper=true,
  colorlinks=true,
  linkcolor=red,
  citecolor=red,
  pdfpagemode=None]{hyperref}
\usepackage{graphicx}
\usepackage{color}

\usepackage{tikz}
\usepackage{pgfplots}
\usepackage{listings}

\newcommand{\bbR}{\mathbb{R}}
\newcommand{\bfr}{\mathbf{r}}
\newcommand{\bfv}{\mathbf{v}}
\newcommand{\bfa}{\mathbf{a}}
\newcommand{\bff}{\mathbf{f}}
\newcommand{\bfg}{\mathbf{g}}
\newcommand{\Wps}{W_{\mathrm{poly6}}}
\newcommand{\Wsp}{W_{\mathrm{spiky}}}
\newcommand{\Wvi}{W_{\mathrm{viscosity}}}

\newcommand{\matlab}{\textsc{Matlab}}

\lstset{language=c,columns=flexible}

\begin{document}

\pagestyle{fancy}
\lhead{Bindel, Spring 2014}
\rhead{Applications of Parallel Computers (CS 5220)}
\fancyfoot{}

\section{The SPH equations}

Smoothed particle hydrodynamics (SPH) is a particle-based method for
simulating the behavior of fluids.  Each computational particle
carries along information about the fluid in a little region, such as
the velocity and density; and during the course of the simulation,
these particles interact with each other in a way that models the
dynamics of a fluid.  In this project, we will tune a simple 3D 
SPH method described by M\"uller et al for use in
graphics~\cite{muller}.  There are better methods for this problem
(and this implementation is arguably incomplete -- we left out the
surface tension forces), but this method does illustrate common issues
in particle-based methods.

Our simulation basically solves a system of ordinary differential
equations\footnote{%
  We describe the derivation of the equations in a separate document.
  It may interest those of you who care about fluid dynamics, but it is
  not critical to understand the derivation in order to do the assignment.
}
for a collection of particles with equal masses $m$ and
interaction radii $h$.  Each particle $i$ has a position $\bfr_i$,
a velocity $\bfv_i$, and a density $\rho_i$.  Particle $i$ interacts
with the set $N_i$ of particles within radius $h$ of $i$.
The density is computed at each step by
\[
  \rho_i = \frac{4m}{\pi h^8} \sum_{j \in N_i} (h^2 - r^2)^3.
\]
The acceleration is computed by the rule
\[
  \bfa_i = \frac{1}{\rho_i} \sum_{j \in N_i} \bff_{ij}^{\mathrm{interact}} + \bfg,
\]
where
\[
  \bff_{ij}^{\mathrm{interact}} =
  \frac{45}{\pi h^5} \frac{m_j}{\rho_j} (1-q_{ij}) \left[
    \frac{k}{2} (\rho_i + \rho_j - 2 \rho_0) \frac{(1-q_{ij})}{q_{ij}} \bfr_{ij} -
    \mu \bfv_{ij}
  \right],
\]
where $\bfr_{ij} = \bfr_i-\bfr_j$, $\bfv_{ij} = \bfv_i-\bfv_j$,
and $q_{ij} = \|\bfr_{ij}\|/h$.  The parameters in these expressions are
\begin{align*}
  \rho_0 &= \mbox{reference mass density} \\
  k      &= \mbox{bulk modulus} \\
  \mu    &= \mbox{viscosity} \\
  \bfg   &= \mbox{gravitational vector}
\end{align*}
By default, we choose most of these parameters to be appropriate to a liquid
like water.  The exception is the bulk modulus, which is chosen so that the
computational speed of sound
\[
  c_s = \sqrt{\frac{k}{\rho_0}}
\]
is large relative to the typical velocities we expect to see in the
simulation, but not too large.  Choosing $k$ to be very large (e.g.~on
the scale of the bulk modulus for water) severely limits the time step
size needed for stable simulation.

\input{codes.tex}

\begin{thebibliography}{10}

\bibitem{muller}
{\sc M. M\"uller, D. Charypar, and M. Gross}.
{\em Particle-based fluid simulation for interactive applications},
in Proceedings of Eurographics/SIGGRAPH Symposium on Computer Animation.

\end{thebibliography}

\end{document}
